\documentclass[12pt]{article}
\usepackage[utf8]{inputenc}
\usepackage{authblk}
\usepackage[english,russian]{babel}
\usepackage{color}
\usepackage{listings}
\usepackage{caption}
\DeclareCaptionFont{white}{\color{white}}
\captionsetup[lstlisting]{format=listing,labelfont=white,textfont=white}

\title{Code Style}
\author{ИА-031 Гарин Н.Е.}
\date{Февраль 2022}
\affil{email: nikgar2002@bk.ru}
\affil{github: gazmanka}

\begin{document}
\lstset{language=C, 
numbers=left,
basicstyle=\small\sffamily

\maketitle

\section{Введение}
Создание Код-Стайла для C++

\section{Основная часть}
\textbf{Используемые языки программирования}
\begin{itemize}
    \item C/C++
\end{itemize}
\textbf{Пробелы и отступы}

\\Операторы и операнды разделяются пробелом:

\begin{lstlisting}
int x = (a + b) * c / d;
\end{lstlisting}

Так же пробелом отделяются и фигурные скобки:

\begin{lstlisting}
if(a == 5) { return; }
\end{lstlisting}
\textbf{Оформление циклов и операторов управления}

При использовании циклов или операторов управления используются отступы и переходы на новую строку, если это нужно:

\begin{lstlisting}
while(1) 
{ 
    if(a == 5) 
    {
        return;
    }
    a++;
}
\end{lstlisting}
\textbf{Разделение функций и блоков кода}

Функции и разные по смыслу блоки кода разделяются пустой строкой:
\begin{lstlisting}
int addFirst(...) 
{
....
}

int addLast(...) 
{
...
}
\end{lstlisting}
\textbf{Названия функций и переменных}

Названия функций и переменных должны быть логичными, не однобуквенными (названия итераторов могут состоять из одной буквы):
\begin{lstlisting}
int i;
int counter;
\end{lstlisting}

Названия функций должны быть записаны в смешанном регистре и начинаться с нижнего:
\begin{lstlisting}
int addFirst(...) 
{
....
}

int addLast(...) 
{
...
}
\end{lstlisting}

Именнованые константы должны быть записаны в верхнем регистре с нижним подчеркивание в качестве разделителя:
\begin{lstlisting}
const int N = 10;
const int MAX_SIZE = 1000;
\end{lstlisting}

\section{Список литературы}

\begin{enumerate}
    \item https://tproger.ru/translations/stanford-cpp-style-guide/
    \item https://habr.com/ru/post/172091/
    \item https://habr.com/ru/company/ruvds/blog/574352/
\end{enumerate}
\end{document}